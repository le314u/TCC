\begin{frame}{Considerações Finais}
    \begin{itemize}
        \item O protótipo apresentado demonstrou uma \textbf{precisão satisfatória} na avaliação do movimento de barra fixa.
    
        \item \textbf{Dificuldade para a verificação do tipo de pegada} (pronada) devido a uma \textbf{imprecisão da ferramenta} em relação a detecção dos pontos de referência 17, 18 , 19 e 20
    
        \item O tempo médio de processamento de um frame foi de 0,2481895403 segundos. No entanto, para o processamento em tempo real de um vídeo com uma taxa de 24 frames por segundo, o tempo limite de processamento é de 0,0416 segundos por frame ou seja gastou\textbf{ 5,96 vezes mais tempo do que o esperado.}
    
        \item As 2 funções mais relevantes para a alta taxa de latência do processamento do frame são \textbf{’verify\_eph’} e \textbf{’verify\_maoBarra’} sendo responsável por cerca de \textbf{85.14\%} do tempo gasto por ’process\_cell’
    
        \item O protótipo produzido tem potencial para crescer e amadurecer;
    \end{itemize}
\end{frame}

\begin{frame}{Trabalhos Futuros}
    \begin{itemize}
        \item Melhoria dos pontos citados ( Execução e pegada pronada)
        \item Desejavel para trabalhos futuros um aprimoramento do relatório de modo que contabilize a quantidade de tempo gasto em cada tipo de contração muscular
        \item Elaborar estratégias para avançar com o grau de maturidade da ferramenta.
    \end{itemize}
\end{frame}
