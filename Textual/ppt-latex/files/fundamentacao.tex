\begin{frame}{Transformada de Hough}
    \begin{itemize}
        \item Uma reta pode ser descrita como:  y = mx + b
        \item Pela parametrização rho = x $∗$ cos(theta) + y $∗$ sin(theta)
        \item Onde o parâmetro rho é a mínima distância da reta a origem, centro do plano, e theta é o ângulo de inclinação da normal à reta, sendo a normal uma linha perpendicular a direção da reta
        \item Dois pontos p e q no plano da imagem definem uma reta pq e correspondem a dois senóides no plano de Hough, a intersecção dos dois senóides representa a reta pq que passa pelos dois pontos no plano da imagem
    \end{itemize}
\end{frame}