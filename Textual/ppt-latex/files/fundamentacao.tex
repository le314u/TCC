\begin{frame}{Transformada de Hough}
    \begin{itemize}
        \item A transformada de Hough é capaz de detectar grupos de pixels que pertencem a uma linha reta (mesmo que esteja quebrada ou com ruídos).
        
        \item Uma reta pode ser descrita como:  y = mx + b
        
        \item Pela parametrização $\rho = x \protect\ast \cos(\theta) + y \protect\ast \sin(\theta)$

        
        \item Onde o parâmetro $\rho$ é a mínima distância da reta a origem, centro do plano, e $\theta$ é o ângulo de inclinação da normal à reta, sendo a normal uma linha perpendicular a direção da reta
        
        \item Dois pontos \textbf{p} e \textbf{q} no plano da imagem definem uma reta \textbf{pq} e correspondem a dois senoides no plano de Hough, a intersecção dos dois senóides representa a reta \textbf{pq} que passa pelos dois pontos no plano da imagem
    \end{itemize}
\end{frame}