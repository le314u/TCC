% ----------------------------------------------------------
\chapter{Introdução}
% ----------------------------------------------------------

É possível encontrar ferramentas computacionais que fazem uso  de visão computacional para auxiliarem no aprimoramento de movimentos de certas atividades físicas \cite{vcBicicleta} \cite{vcFutebol} \cite{futebolTatica}. Como por exemplo o BFit4All que por meio da análise de fotografias do ciclista durante o movimento de pedalada propõe  sugestões de ajustes a serem feitos na bicicleta, contribuindo assim com a saúde e desempenho do mesmo  \cite{vcBicicleta}. Contudo, por questão de investimento/retorno financeiro ou outra justificativa ainda existem poucas ferramentas computacionais analíticas com o proposito de aprimorar exercícios requisitados, durante o Teste de Aptidão Física (TAF).

O TAF é uma das etapas comuns em concursos públicos para provimento de cargos em carreiras relacionadas a área de segurança pública. Sua finalidade é atestar a capacidade do indivíduo em desempenhar funções específicas do cargo por meio de parâmetros pré estabelecidos em edital.Os exercicios avaliados comumemnte são : barra fixa; salto de impulsão horizontal; corrida de 12km ou 2.400m; natação; flexão abdominal. Entretanto, esta é uma etapa muito negligenciada e acaba sendo responsável por um alto índice de reprovação \cite{reprovaTAF}.


De acordo com o Governo da Paraíba o exercício de barra fixa foi o maior responsavel por reprovação durante o TAF dos concursos  da Polícia Militar da Paraiba (PMPB) e Corpo de Bombeiros Militar da Paraíba (PMPB) no ano de 2018 \cite{barraTAF}.O movimento de barra fixa ou flexão de braços na barra fixa é um exercício que avalia a força e a resistência muscular dinâmica dos músculos do tronco e dos membros superiores. É um teste amplamente empregado em campo, devido à facilidade de aplicação, baixo custo e alta reprodutibilidade\cite{barraFixa}. 

%Os principais músculos superficiais usados na sua execução são: latíssimo do dorso; bíceps braquial; braquiorradial e peitoral maior esternocostal\cite{cinesiologiaBarraFixa}. 


Portanto, este trabalho visa a criação de uma ferramenta que auxilie na melhoria do movimento de barra fixa, focado na realização do TAF de concursos publicos da area militar, avaliando a correta execução do movimento de acordo com parametros pre estabelecidos em editais por meio da visão computacional e redes neurais.

% ----------------------------------------------------------
\section{Justificativa}
% ----------------------------------------------------------

O valor social deste trabalho se destaca  pois atua como ferramenta auxiliar na preparação para etapa de TAF de concursos publicos, levando em consideração que o salario minimo em 2022 é de R\$ 1.212,00 (um mil duzentos e doze reais) \cite{salarioMin} o custo de contratação de um personal trainer é em media R\$60,00 (sessenta reais) hora/aula \cite{valorPersonal}  ou seja cerca de 4,95\% do salário mínimo por aula,  esta ferramenta servirá de apoio para reduzir o custo de treinamento e assim podendo contribuir para uma igualdade de condições para o preparo do TAF.

Este trabalho visa criar uma condição de analise extremamente objetiva para o movimento de Barra fixa, reduzindo ao maximo os criterios de subjetividade.

Além de seu valor social este trabalho é responsavel por consolidar possíveis aplicações da visão computacional alinhadas a área do treinamento funcional em específico pelo fato de atuar como uma ferramenta que extrai informações sobre biomecanica apartir de uma série de imagens e faz algumas tomadas de decisões analisando conceitos como insuficiencia ativa e insuficiência passiva.




















% ----------------------------------------------------------
\section{Objetivos}
% ----------------------------------------------------------

\subsection{Objetivo Geral}	

Projetar e desenvolver uma ferramenta de visão computacional para análise biomecânica do bíceps braquial e braquiorradial durante o movimento de barra fixa.

\subsection{Objetivos Específicos}	

\begin{itemize}

    \item Projetar uma ferramenta de visão computacional (em nível de prova de conceito) que analise o movimento de barra fixa durante o treinamento de um indivíduo e possibilite o aperfeiçoamento do movimento por meio de sugestões ao usuário sobre a ocorrência de insuficiência ativa e passiva no bíceps braquial

    \item Aplicar uma metodologia já existente para análise de insuficiência ativa ou passiva no músculo bíceps braquial durante o movimento de barra fixa por meio de uma aplicação da visão computacional.

    \item Desenvolver uma ferramenta para análise, com o uso da visão computacional; para indicar uma insuficiência ativa ou passiva no músculo bíceps braquial durante o movimento de barra fixa.

\end{itemize}