%%%%%%%%%%%%%%%%%%%%%%%%%%%%%%%%%%%%%%%%%%%%%%%%%%%%%%%%%%%%%%%%%
%
%   Template para monografia de TCC-2018 - versao 1.0.alfa
%
%%%%%%%%%%%%%%%%%%%%%%%%%%%%%%%%%%%%%%%%%%%%%%%%%%%%%%%%%%%%%%%%%
%%
%% Este template utiliza o modelo mantido pela abnt e foi baseado 
%% no abtex2-modelo-trabalho-academico.tex, v-1.9.6 laurocesar
%% Copyright 2012-2016 by abnTeX2 group at http://www.abntex.net.br/ 
%%
% -------------------------------------------------------------------
%% This work may be distributed and/or modified under the conditions 
%% of the LaTeX Project Public License, either version 1.3 of this 
%% license or (at your option) any later version. The latest version 
%% of this license is in http://www.latex-project.org/lppl.txt
%% and version 1.3 or later is part of all distributions of LaTeX
%% version 2005/12/01 or later.
%%
%% This work has the LPPL maintenance status `maintained'.
%% 
%% The Current Maintainer of this work is the abnTeX2 team, led
%% by Lauro César Araujo. Further information are available on 
%% http://www.abntex.net.br/
%%
%% This work consists of the files abntex2-modelo-trabalho-academico.tex,
%% abntex2-modelo-include-comandos and abntex2-modelo-references.bib
% -------------------------------------------------------------------
%%
%% Modelo de Monografia em conformidade com ABNT NBR 14724:2011
%%
%%%%%%%%%%%%%%%%%%%%%%%%%%%%%%%%%%%%%%%%%%%%%%%%%%%%%%%%%%%%%%%%%
%%
%% ATENÇAO para as opções:
%%
%% para modo rascunho, sem páginas brancas, use: 
%% openany, oneside, nopartblankpage
%%
%% para modo normal, com tudo que ABNT exige, use:  
%% openright, twoside, partpageblank
%%

\documentclass[
	% -- opções da classe memoir --
	12pt,			% tamanho da fonte
	openany,		% capítulos começam em qq página (isso elimina várias pág brancas)
	%openright,		% capítulos começam em pág ímpar (insere página vazia caso preciso)
	oneside,		% considera impressão de um só lado (gera menos pág brancas)
	%twoside,		% para impressão em recto e verso. Oposto a oneside
	a4paper,		% tamanho do papel. 
	% -- opções da classe abntex2 --
	%chapter=TITLE,		% títulos de capítulos convertidos em letras maiúsculas
	%section=TITLE,		% títulos de seções convertidos em letras maiúsculas
	%subsection=TITLE,	% títulos de subseções convertidos em letras maiúsculas
	%subsubsection=TITLE,% títulos de subsubseções convertidos em letras maiúsculas
	% -- opções do pacote babel --
	english,		% idioma adicional para hifenização
	%french,		% idioma adicional para hifenização
	%spanish,		% idioma adicional para hifenização
	%portuges		% o último idioma é o principal do documento
	brazil			% o último idioma é o principal do documento
	]{abntex2}


% ------------------------------------------------------------------------
%   Componente do template para monografia de TCC-2018 - versao 1.0.alfa
% ------------------------------------------------------------------------

%------------ Pacotes básicos 
\usepackage{lmodern}			% Usa a fonte Latin Modern			
\usepackage[T1]{fontenc}		% Selecao de codigos de fonte.
\usepackage[utf8]{inputenc}		% Codificacao do documento (conversão automática dos acentos)
\usepackage{lastpage}			% Usado pela Ficha catalográfica
\usepackage{indentfirst}		% Indenta o primeiro parágrafo de cada seção.
\usepackage{color}			% Controle das cores
\usepackage{graphicx}			% Inclusão de gráficos
\usepackage{microtype} 			% para melhorias de justificação
		
%------------ Pacotes adicionais
\usepackage{blindtext}			% para geração de dummy text
\usepackage[table,xcdraw]{xcolor}
\usepackage{acronym}
\usepackage{amsmath}
\usepackage{listings}
\usepackage{float}



%------------ Pacotes de citações
\usepackage[brazilian,hyperpageref]{backref}	% Paginas com as citações na bibl
\usepackage[alf]{abntex2cite}		% Citações padrão ABNT
\usepackage{pdfpages}			% Saida em pdf

%------------ Comandos úteis
\newcommand{\aspas}[1]{``#1''}
\newcommand{\sizeImg}{1}

% ----------- CONFIGURAÇÕES DE PACOTES

% Configurações do pacote backref
% Usado sem a opção hyperpageref de backref
\renewcommand{\backrefpagesname}{Citado na(s) página(s):~}
% Texto padrão antes do número das páginas
\renewcommand{\backref}{}
% Define os textos da citação
\renewcommand*{\backrefalt}[4]{
	\ifcase #1 %
		Nenhuma citação no texto.%
	\or
		Citado na página #2.%
	\else
		Citado #1 vezes nas páginas #2.%
	\fi}%
% ---

% Espaçamentos entre linhas e parágrafos 
% O tamanho do parágrafo é dado por:
\setlength{\parindent}{1.3cm}
% Controle do espaçamento entre um parágrafo e outro:
\setlength{\parskip}{0.2cm}  % tente também \onelineskip

% compila o indice
\makeindex

% ----------------------------------------------------
% Configurações de aparência do PDF final
% ----------------------------------------------------

% alterando o aspecto da cor azul
\definecolor{blue}{RGB}{41,5,195}

% informações do PDF
\makeatletter
\hypersetup{
     	%pagebackref=true,
		pdftitle={\@title}, 
		pdfauthor={\@author},
    	pdfsubject={\imprimirpreambulo},
	    pdfcreator={LaTeX with abnTeX2},
		pdfkeywords={abnt}{latex}{abntex}{abntex2}{trabalho acadêmico}, 
		colorlinks=true,       		% false: boxed links; true: colored links
    	linkcolor=blue,          	% color of internal links
    	citecolor=blue,        		% color of links to bibliography
    	filecolor=magenta,      		% color of file links
		urlcolor=blue,
		bookmarksdepth=4
}
\makeatother

% ----------------------------------------------------
% Comandos para geração de itens textuais 
% ----------------------------------------------------

\newcommand{\imprimirfichacatalografica}{

\begin{fichacatalografica}
	\sffamily
	\vspace*{\fill}					% Posição vertical
	\begin{center}					% Minipage Centralizado
	\fbox{\begin{minipage}[c][10cm]{14.5cm}		% Largura
	\small
	
	\hspace{1cm} \imprimirautor\\

	\begingroup
        \leftskip4em
        \rightskip\leftskip
	
	\imprimirtitulo  / \imprimirautor. -- \imprimirlocal, \imprimirdata.
	
	\pageref{LastPage} p. : il.\\
	
	\imprimirorientadorRotulo~\imprimirorientador\\
	
	%\imprimirtipotrabalho~--~\imprimirinstituicao, \imprimirdata.\\
	Monografia para trabalho de conclusão de curso (graduação)~--~Instituto Federal 
	de Educação, Ciência e Tecnologia de Minas Gerais, Ciência da Computação, 
	Formiga, \imprimirdata.\\
	
	1. Visão computacional.
	2. Barra fixa.
	3. TAF.
	I. \imprimirorientador.
	II. Instituto Federal de Educação, Ciência e Tecnologia de Minas Gerais.
	III. Ciência da Computação.
	IV. \imprimirtitulo 			
        \par
        \endgroup
	\end{minipage}}
	\end{center}
\end{fichacatalografica}

}

\newcommand{\imprimirerrata}{

\begin{errata}
Exemplo de errata\\[1cm]

FERRIGNO, C. R. A. \textbf{Tratamento de neoplasias ósseas apendiculares com
reimplantação de enxerto ósseo autólogo autoclavado associado ao plasma
rico em plaquetas}: estudo crítico na cirurgia de preservação de membro em
cães. 2011. 128 f. Tese (Livre-Docência) - Faculdade de Medicina Veterinária e
Zootecnia, Universidade de São Paulo, São Paulo, 2011.

\begin{table}[htb]
\center
\footnotesize
\begin{tabular}{|p{1.4cm}|p{1cm}|p{3cm}|p{3cm}|}
  \hline
   \textbf{Folha} & \textbf{Linha}  & \textbf{Onde se lê}  & \textbf{Leia-se}  \\
    \hline
    1 & 10 & auto-conclavo & autoconclavo\\
   \hline
\end{tabular}
\end{table}

\end{errata}

}

\newcommand{\imprimirfolhadeaprovacao}[1]{

\begin{folhadeaprovacao}
  \begin{center}
   {\ABNTEXchapterfont\large\imprimirautor}

   \vspace*{\fill}\vspace*{\fill}
   \begin{center}
     \ABNTEXchapterfont\bfseries\Large\imprimirtitulo
   \end{center}
   \vspace*{\fill}
    
   \hspace{.45\textwidth}
   \begin{minipage}{.5\textwidth}
       \imprimirpreambulo
   \end{minipage}%
   \vspace*{\fill}        
   \end{center}
   
   Trabalho aprovado em {#1}.
   
   \vspace{1cm}
   \hspace{4cm} BANCA EXAMINADORA    

   \assinatura{\textbf{\imprimirorientador} \\ Orientador} 
   \assinatura{\textbf{Fulano} \\ Convidado 1}
   \assinatura{\textbf{Sicrano} \\ Convidado 2}
   %\assinatura{\textbf{Professor} } %\\ Convidado 3}
   %\assinatura{\textbf{Professor} } %\\ Convidado 4}
      
   \begin{center}
    \vspace*{0.5cm}
    {\large\imprimirlocal}
    \par
    {\large\imprimirdata}
    \vspace*{1cm}
  \end{center}
  
\end{folhadeaprovacao}

}

	

\graphicspath{{./figuras/}}   	% pasta contendo todas as figuras

\nopartblankpage  		% elimina páginas em branco
% \partpageblank		% permite páginas em branco

% ----------------------------------------------------
% Informações de dados para CAPA e FOLHA DE ROSTO
% ----------------------------------------------------

\titulo{Título do TCC}
\autor{Lucas Mateus Fernandes}
\local{Formiga - MG}
\data{2023}
\orientador{Fernando Paim Lima}
% \coorientador{Felipe Augusto Lima Reis}
\instituicao{%
  Instituto Federal de Educação, Ciência e Tecnologia de Minas Gerais \par
  Campus Formiga \par
  Ciência da Computação
  }
\tipotrabalho{Monografia}

% O preambulo deve conter o tipo do trabalho, o objetivo, o nome da instituição e a área de concentração 
%\preambulo{Monografia do trabalho de conclusão de curso apresentado ao Instituto Federal Minas Gerais - Campus Formiga, como requisito parcial para a obtenção do título de Bacharel em Ciência da Computação.}


\preambulo{Proposta de projeto do Trabalho de Conclusão de Curso do Curso de Bacharelado em Ciência da Computação, IFMG – Campus Formiga.}

    
% ----------------------------------------------------
% INÍCIO DO DOCUMENTO
% ----------------------------------------------------

\begin{document}

%\selectlanguage{portuges}
\selectlanguage{brazil}

\frenchspacing  % retira espaço extra obsoleto entre as frases.

%-----------------------------------------------------
% ELEMENTOS PRÉ-TEXTUAIS
%-----------------------------------------------------
% \pretextual

\imprimircapa

\imprimirfolhaderosto*  % o comando com * indica que haverá ficha bibliográfica

%------------- Ficha catalográfica
% São os ``Dados internacionais de catalogação-na-publicação''.
% Escolha uma das opções: usar página pdf pronta (gerada pela biblioteca?!)
% ou gerar a página com o comando \imprimirfichacatalografica. 
% (Mais detalhes no preludio e documentacao do pacote abntex2.)
%
% \begin{fichacatalografica}
%     \includepdf{ficha_catalografica.pdf}
% \end{fichacatalografica}
\imprimirfichacatalografica

%------------ Errata (se houver)
% Modifique o comando \imprimirerrata no preludio e use esse comando
%\imprimirerrata

%------------ Folha de aprovação
% É um elemento obrigatório da NBR 14724/2011 (seção 4.2.1.3). 
% Você pode utilizar a página modelo até a aprovação do trabalho. 
% Após isso, gere uma página com a imagem da folha assinada pela banca,
% comente o comando \imprimirfolhadeaprovacao e use o \includepdf 
\imprimirfolhadeaprovacao{06 de junho de 2018}
% \includepdf{folhadeaprovacao_final.pdf}

%------------ Dedicatória
\begin{dedicatoria}
   \vspace*{\fill} \centering \noindent 
   \textit{ Este trabalho é dedicado a tudo e todos, exceto os cachorros de rua.} 
   \vspace*{\fill}
\end{dedicatoria}

%------------ Agradecimentos
\begin{agradecimentos}
  Gostaria de agradecer meus pais por tudo que fizeram e
  aos meus professores pela paciência que tiveram.
\end{agradecimentos}


%------------ Epígrafe
\begin{epigrafe}
   \vspace*{\fill}
   \begin{flushright}
	\textit{\aspas{Nunca atribua à malícia/maldade o que pode ser adequadamente explicado pela estupidez.}
	(Navalha de Hanlon) }
   \end{flushright}
\end{epigrafe}

%------------ RESUMOS
\setlength{\absparsep}{18pt} % ajusta o espaçamento dos parágrafos do resumo

% resumo em português
\begin{resumo}
 A escrita da monografia é um problema clássico no Trabalho de Conclusão de Curso (TCC).
 Aqui é o lugar onde o autor explica brevemente o conteúdo do seu trabalho. Para usar este
 template é necessário instalar alguns pacotes do LaTeX, particularmente o abntex2. O pacote
 blindtex não é realmente necessário, serve apenas para gerar lero-lero neste exemplo.
 
 \textbf{Palavras-chave}: Monografia, LaTeX, TCC.
\end{resumo}

% resumo em inglês
\begin{resumo}[Abstract]
 \begin{otherlanguage*}{english}
  The writing of the monograph is a classic problem in the Course Conclusion Work.
  Here is where the author briefly explains the content of his work.
 
 \textbf{Keywords}: Monograph, LaTeX.
 \end{otherlanguage*}
\end{resumo}

% Consulte o manual da classe abntex2 para maiores 
% orientações sobre os seguintes tópicos:

%------------ Lista de ilustrações
\pdfbookmark[0]{\listfigurename}{lof}
\listoffigures*
\cleardoublepage

%------------ Lista de tabelas

%------------ Lista de abreviaturas e siglas

%------------ Lista de símbolos

%------------ Sumario
\pdfbookmark[0]{\contentsname}{toc}
\tableofcontents*
\cleardoublepage

% ----------------------------------------------------------
% ELEMENTOS TEXTUAIS
% ----------------------------------------------------------
\textual
% ----------------------------------------------------------
\chapter{Introdução}
% ----------------------------------------------------------

É atípico encontrar ferramentas analíticas tecnológicas que auxiliem na melhoria ou acusem a execução correta do movimento de barra fixa. Mesmo que seja possível encontrar ferramentas computacionais que fazem uso  do processamento de imagens para auxiliarem no aprimoramento de movimentos em diversas atividades físicas \cite{vcBicicleta} \cite{vcFutebol} \cite{futebolTatica}. Todavia, por questão de investimento/retorno financeiro ou outra justificativa, ainda existem poucas ferramentas computacionais analíticas com o proposito de aprimorar exercícios requisitados, durante o Teste de Aptidão Física (TAF).

O TAF é uma das etapas comuns em concursos públicos para provimento de cargos em carreiras relacionadas a área de segurança pública, sua finalidade é atestar a capacidade do indivíduo em desempenhar funções específicas do cargo por meio de parâmetros pré estabelecidos em edital. Os exercicios avaliados comumemnte são: barra fixa; salto de impulsão horizontal; corrida de 12km ou 2.400m; natação; flexão abdominal. Entretanto, esta é uma etapa muito negligenciada e acaba sendo responsável por um alto índice de reprovação \cite{reprovaTAF}.

De acordo com o Governo da Paraíba o exercício de barra fixa foi o maior responsavel por reprovação durante o TAF dos concursos  da Polícia Militar da Paraiba (PMPB) e Corpo de Bombeiros Militar da Paraíba (PMPB) no ano de 2018 \cite{barraTAF}.O movimento de barra fixa ou flexão de braços na barra fixa é um exercício que avalia a força e a resistência muscular dinâmica dos músculos do tronco e dos membros superiores. É um teste amplamente empregado em campo, devido à facilidade de aplicação, baixo custo e alta reprodutibilidade\cite{barraFixa}. 

Portanto, este trabalho visa a criação de uma ferramenta que auxilie na melhoria do movimento de barra fixa, focado na realização do TAF de concursos publicos da area militar, avaliando a correta execução do movimento de acordo com parametros pre estabelecidos em editais por meio da visão computacional e redes neurais.

% ----------------------------------------------------------
\section{Justificativa}
% ----------------------------------------------------------


O concurso público é um instrumento voltado para a efetivação dos princípios da impessoalidade e da isonomia no acesso aos cargos públicos \cite{} (art. 37, da Constituição da República Federativa do Brasil), porem a isonomia de acesso não garante uma igualdade de condiçoes levando em consideração que o salario minimo em 2022 é de R\$ 1.212,00 (um mil duzentos e doze reais) \cite{salarioMin} o custo de contratação de um personal trainer é em media R\$60,00 (sessenta reais) hora/aula \cite{valorPersonal} ou seja cerca de 4,95\% do salário mínimo por aula, tal ferramenta pode se tornar uma alternativa barata para a democratização do acesso a um treino minimamente eficiente, podendo aumentar a  igualdade de condiçoes relacionadas ao treinamento de barra fixa.


O uso da visão computacional associado ao movimento de barra fixa usado nos TAF's pode criar uma condição de analise objetiva sobre a execução correta do movimento de Barra fixa, reduzindo ao maximo os criterios de subjetividade pois mesmo o TAF tendo criterios pre estabelecidos em edital, no dia do exame as tomada de decisões por parte dos avaliadores poderão estar relacionadas ao cansaço humano devido uma fadiga de decisão\cite{fadiga}.


A conciencia comporal é construida por meio da percepção durante o execício a fim de poder aprimorar o movimento e o domínio do corpo \cite{consciencia}, portanto a percepção do posicionamento do corpo em relação ao espaço e as partes ou segmentos do corpo entre si são essenciais para o processo de formação da conciencia comporal. Portanto essa ferramenta tem como foco extrair informações biomecanicas apartir de uma série de imagens e dar um retorno ao individuo sobre a amplitude correta do movimento, tempo de contração, momento que o musculo entra em insuficiencia ativa ou insuficiência passiva fazendo assim uma ferramenta auxiliar no processo de formação da conciencia corporal.


Portanto essa ferramenta tem como foco extrair informações apartir de uma série de imagens da execução do moviemnt ode barra fixa e dar um retorno ao individuo sobre a validação do movimento, amplitude do movimento, tempo de contração, momento que o musculo entra em insuficiencia ativa ou insuficiência passiva fazendo assim uma ferramenta util para o treinamento de barra fixa associada ao TAF alem de auxiliar no processo de formação da conciencia corporal 




% ----------------------------------------------------------
\section{Objetivos}
% ----------------------------------------------------------

\subsection{Objetivo Geral}	

Projetar e desenvolver uma ferramenta de visão computacional para análise do movimento de barra fixa associada ao TAF.

\subsection{Objetivos Específicos}	

\begin{itemize}

    \item Projetar uma ferramenta de visão computacional (em nível de prova de conceito) que analise o movimento de barra fixa.

    \item Projetar uma ferramente, com o uso da visão computacional, que auxilie na formação da conciência do corporal do individuo por meio de feedback ativo durante o movimento.

    \item Aplicar uma metodologia já existente para análise de insuficiência ativa ou passiva no músculo bíceps braquial durante o movimento de barra fixa por meio de uma aplicação da visão computacional.

    \item Usar a visão computacional para verificar a corretude do movimento de barra fixa tendo como parametros alguns editais de concursos publicos da esfera militar.
    
    \item Desenvolver uma ferramenta, com o uso da visão computacional, para indicar uma insuficiência ativa ou passiva no músculo bíceps braquial durante o movimento de barra fixa.


\end{itemize}
% ----------------------------------------------------------
\chapter{Fundamentação Teórica}
% ----------------------------------------------------------

Neste capítulo são mostrados os principais fundamentos necessários para entender os
conceitos abordados no presente trabalho.

\section[TAF]{Teste de Aptidão Física}


O Teste de Aptidão Física (TAF). é uma etapa comum em concursos públicos para provimento de cargos em carreiras relacionadas a área de segurança pública. Seu objetivo é atestar a capacidade do indivíduo em desempenhar funções específicas do cargo por meio da avaliação de determinados exercícios. A avaliação é embasada em exercícios e parâmetros pré estabelecidos em edital como por exemplo: quantidade de repetições de um determinado movimento para o exercício de barra fixa ou flexão abdominal, tempo gasto para translocação em uma determinada distancia nas corridas de 12km, 2.400m ou natação, ou até mesmo a distância percorrida durante o salto de impulsão horizontal.


\section[Barra Fixa]{Barra Fixa}

Nos ultimos 5 anos é possivel ver uma série de concursos publicos que avaliam o movimento de barra fixa no teste de aptidão física (TAF) como por exemplo:
Concurso público para admissão ao curso de formação de soldados bombeiros militar do quadro de praças (qp-bm) e do quadro de praças especialistas – (qpe-bm) do corpo de bombeiros militar de minas gerais para o ano de 2020.\cite{eCBMG2018};
Concurso público para admissão ao curso de formação de soldados do quadro de praças especialistas da polícia militar de minas gerais.\cite{ePMMG2021};
Concurso público para o provimento de vagas nos cargos de delegado de polícia federal, agente de polícia federal, escrivão de polícia federal e papiloscopista policial federal.\cite{ePF2021};
Concurso público para provimento de cargos da carreira de agente de segurança penitenciário/policial penal do quadro de pessoal da secretaria de estado de justiça e segurança\cite{ePP2021}.

O teste de flexão em barra fixa ou teste dinâmico de barra fixa sofre algumas divergencias de acordo com o edital, porem os principios do movimento são semelhantes:

A barra fixa é instalada a uma altura tal, que o avaliado, mantendo-se pendurado, com os cotovelos em extensão, não tenha contato dos pés com o solo;
A posição da pegada é pronada (dorso da mão voltado para o rosto) e a abertura das mãos corresponde à distância biacromial (largura dos ombros);
Após assumir essa posição, o avaliado deverá elevar o corpo até que o queixo ultrapasse o nível da barra, após o que retornará à posição inicial;
O movimento é repetido tantas vezes quanto possível, sem limite de tempo.
Os cotovelos deverão estar em extensão total para o início de flexão;
É permitido repouso entre um movimento e outro, contudo, o avaliado não poderá tocar os pés no solo;
Não são permitidos movimentos de quadris ou pernas e extensão da coluna cervical como formas de auxiliar na execução da prova.
Somente é contado o número de movimentos completados corretamente.

\begin{figure}[!htb]
	\centering
	\includegraphics[scale=0.45]{tostines.jpg}
	\caption{Exemplo de objeto causador de Efeito Tostines}
	\label{fig-tostines}
\end{figure}


\section[Visão computacional]{Visão computacional}
Visão computacional é a ciência que estuda e desenvolve tecnologias que permitem extrair características de imagens capturadas por diferentes tipos de sensores e então permitem reconhecer, manipular e processar dados sobre os objetos que compõem a imagem capturada \cite{VisaoComp} \cite{visao}.


\section[Visão computacional]{Estimativa de pose humana}
Estimativa de Pose Humana (EPH) ou Identificação de Pose é um problema geral em Visão Computacional, que tem o objetivo de detectar a posição e a orientação de uma pessoa  em imagens ou vídeos prevendo a localização de alguns pontos-chave específicos como  cabeça, ombros, cotovelos, mãos, quadril, joelhos e pés.\cite{edh}

EPH pode ser bidimensional ou tridimensional, sendo o primeiro responsabel por estimar as cordenadas X Y e o segundo as cordenadas X Y Z de um espaço virtual que geralmente é inferido a partir de uma imagem bidimensional. A EDH pode tambem ser classificada de acordo com a quantidade de pessoas a serem estimadas podendo ser uma unica pessoa ou de várias pessoas o que interfere nas resoluções possiveis pois a estimativa de postura de várias pessoas é ligeiramente mais difícil do que o caso de uma única pessoa.\cite{edhDeep}


\section[Visão computacional]{Machine learning}

Machine learning é um ramo da inteligência artificial (IA) focado na análise de dados, que por meio de métodos estatísticos de aprendizado e otimização  permite computadores analisarem conjuntos de dados e identificar padrões e tendências históricas para prever modelos futuros.\cite{ml}

O algoritmo típico de aprendizado de máquina supervisionado consiste em aproximadamente três componentes: Decisão; Erro ; Otimização.

A decisão esta relacionada a uma predição ou classificação com base em alguns dados de entrada, que podem ser rotulados ou não rotulados. O errro avalia a qualidade da predição ou classificação por meio de comparaçoes com o rotulo ou exemplos conhecidos. A otimização de modelo esta relacionada a ajustes para minimizar a diferença entre o exemplo conhecido e a estimativa do modelo.



\section[Visão computacional]{Tensor flow}
O TensorFlow é uma plataforma completa que facilita a criação e a implantação de modelos de Machine learning (ML).



\section[Visão computacional]{OpenCV}
OpenCV (Open Source Computer Vision Library), originalmente, desenvolvida pela Intel, em 2000, é uma biblioteca multiplataforma, totalmente livre ao uso acadêmico e comercial, para o desenvolvimento de aplicativos na área de Visão computacional \cite{}




\section[Articulação]{Articulação}
A articulação é a conexão de duas ou mais superfícies ósseas que promovem movimento \cite{articulacao}

\section[Parte Distal]{Parte Distal}
A parte distal é o ponto mais afastado do tronco ou do ponto de origem \cite{distal}.

\section[Origem Muscular]{Origem Muscular}
Origem é a extremidade do músculo que esta fixada a um ponto fixo ou a uma peça ossea que não se desloca \cite{sisMuscular}.

\section[Inserção Muscular]{Inserção Muscular}
Inserção é a extremidade do músculo que esta fixada a um ponto movel ou a uma peça óssea que se desloca \cite{sisMuscular}.

\section[Cadeia Cinemática]{Cadeia Cinemática}
Os termos cadeia cinemática aberta ou fechada, são usados para demonstrar se o segmento mais distal da cadeia está fixado a algum objeto imóvel ou até mesmo o solo.
Sendo que cadeia cinemática aberta representa uma situação em que a extremidade do membro não está fixada ao solo ou a algum objeto imóvel e assim, o segmento está livre para se mover e cadeia cinemática fechada representa uma situação em que o segmento mais distal da cadeia está fixo ao solo ou a um objeto imóvel \cite{silva2015cinesiologia}.
% ----------------------------------------------------------
\chapter{Tecnologias e Métodos}
% ----------------------------------------------------------
Neste capítulo são apresentadas a metodologia adotada, juntamente com as ferramentas, bibliotecas e bases de dados que serão utilizadas no desenvolvimento desse trabalho

\section[Metodologia]{Metodologia}


 \begin{itemize}
   \item Realização de revisão da literatura abordando o uso de redes neurais no reconhecimento de pose detection.  
   \item Escolha de base Line ?.
   \item Escolha de uma ferramentas de posedetection para utilização.
   \item Criação das regras de negocio para detecção da execução correta da barra fixa.
   \item Desenvolvimento de uma Ferramenta comoputacional para detecção da execução correta da barra fixa.
   \item  Avaliação dos resultados.
 \end{itemize}


\section[Tecnologias]{Tecnologias}

% ----------------------------------------------------------
\chapter{Conclusão}
% ----------------------------------------------------------

\blindtext

\blindtext[2]


% ----------------------------------------------------------
\chapter{Trabalhos futuros}
% ----------------------------------------------------------

\blindtext

\blindenumerate[4]

\blindtext


\phantompart

% ----------------------------------------------------------
% ELEMENTOS PÓS-TEXTUAIS
% ----------------------------------------------------------
\postextual

% Referências bibliográficas
\bibliography{monografia}

% Consulte o manual da classe abntex2 para orientações 
% sobre os seguintes tópicos opcionais:

%------------ Glossário

%------------ Apêndices

%------------ Anexos

% INDICE REMISSIVO
\phantompart
\printindex

\end{document}
\grid
