\section{Justificativa}

O uso da visão computacional associado ao movimento de barra fixa pode proporcionar uma análise mais objetiva da execução correta do exercício, minimizando a subjetividade na avaliação. Embora os critérios de avaliação do \ac{TAF} sejam preestabelecidos nos editais, as decisões tomadas pelos avaliadores no dia do exame podem estar sujeitas à fadiga de decisão \cite{fadiga}. Portanto, a aplicação dessa tecnologia pode ser uma alternativa para reduzir os efeitos da subjetividade na avaliação, aumentando a confiabilidade e a justiça do processo seletivo.


A consciência corporal é construída por meio da percepção do individuo durante o execício a fim de poder aprimorar o movimento e o domínio do corpo \cite{consciencia},  portanto a percepção do posicionamento do corpo em relação ao espaço e as partes ou segmentos do corpo entre si são essenciais para o processo de formação da conciência corporal. Assim o uso dessa ferramenta pode ser útil para auxiliar na construção  da consciência corporal, uma vez que seu foco é extrair informações biomecânicas a partir de uma série de imagens e fornecer um retorno ao indivíduo sobre a execução correta do movimento. Com esses dados em mãos, o indivíduo poderá aprimorar seu movimento e dominar seu corpo, tomando consciência do próprio corpo em relação ao movimento de barra fixa, aprimorando, dessa forma, a execução consciente do movimento de barra fixa.



Deste modo, a ferramenta proposta não apenas se apresenta como uma opção para o treinamento de barra fixa associado ao \ac{TAF}, mas também tem o potencial de contribuir para a evolução do indivíduo na maturidade da consciência corporal.

