% ----------------------------------------------------------
\chapter{Introdução}
% ----------------------------------------------------------


O \ac{TAF} é uma das etapas comuns em concursos públicos para provimento de cargos em carreiras relacionadas à área de segurança pública. Sua finalidade é atestar a capacidade do indivíduo em desempenhar funções específicas do cargo, por meio de parâmetros preestabelecidos em edital. Os exercícios avaliados comumente são: barra fixa, salto de impulsão horizontal, corrida de 12 km ou 2.400 m, natação e flexão abdominal. Entretanto, esta é uma etapa muito negligenciada e acaba sendo responsável por um alto índice de reprovação \cite{reprovaTAF}.

Embora seja possível encontrar ferramentas computacionais que utilizam o processamento de imagens para auxiliar no aprimoramento de movimentos em diversas atividades físicas, como mostrado no trabalho de Franke (\citeyear{vcBicicleta}), Pádua (\citeyear{vcFutebol}) e Paulichen (\citeyear{futebolTatica}), é incomum encontrar ferramentas analíticas tecnológicas que auxiliem na melhoria ou indiquem a execução correta do movimento de barra fixa. No entanto, ainda existem poucas ferramentas computacionais analíticas que visam aprimorar exercícios exigidos durante o \ac{TAF}, possivelmente devido a questões de investimento/retorno financeiro ou outras justificativas.

A flexão de braços na barra fixa, também conhecida como movimento de barra fixa, é uma atividade física que tem como objetivo avaliar a resistência e a força dos músculos do tronco e dos membros superiores. Esse exercício é amplamente utilizado em diversos contextos, principalmente por sua facilidade de aplicação, baixo custo e alta reprodutibilidade \cite{barraFixa}. De acordo com o Governo da Paraíba, a realização do movimento de barra fixa foi identificada como o principal motivo de reprovação durante o \ac{TAF} nos concursos de 2018 para a Polícia Militar da Paraíba (PMPB) e o Corpo de Bombeiros Militar da Paraíba (CBMPB) \cite{barraTAF}.

Portanto, este trabalho visa à criação de uma ferramenta que contribua para o aperfeiçoamento do movimento de barra fixa, proporcionando um feedback visual com informações da análise autônoma da angulação dos cotovelos e dos quadris, quantidade de barras executadas e o atendimento aos critérios estabelecidos nos \ac{TAF}s.