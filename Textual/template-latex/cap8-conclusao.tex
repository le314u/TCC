\chapter{Conclusão}

Neste trabalho, foi desenvolvida e apresentada uma ferramenta a nível de prova de conceito de um sistema de monitoramento do movimento de barra fixa baseado em testes de aptidão física com base em visão computacional, que visa fornecer ao indivíduo um feedback visual sobre o movimento a fim de auxiliar na formação da consciência corporal quanto à corretude do movimento.


\section[Pontos fracos e possíveis melhorias]{Pontos fracos e possíveis melhorias}\label{sec:Pontos fracos e possiveis melhorias}

A ferramenta apresentada demonstra uma precisão satisfatória na avaliação do movimento de barra fixa, de acordo com a maioria dos parâmetros estabelecidos nos editais citados em \ref{sec:Barra Fixa}. No entanto, uma exceção notável diz respeito à análise da pegada da mão na posição pronada. A tentativa de verificar se a componente $x$ da posição do mindinho da mão direita é menor do que a do indicador da mesma mão, e se a componente $x$ da posição do mindinho da mão esquerda é maior do que a do indicador da mesma mão, revelou-se ineficiente, isso ocorreu devido a uma imprecisão da ferramenta MidiaPipe na detecção dos pontos de referência 17, 18 , 19 e 20 referênciados em \ref{fig:Pontos de referencia usados na aplicacao}, fazendo com que em um determinado momento do vídeo, houvesse uma inversão nas referências, o que levou a uma interpretação errônea da pegada.

O tempo médio de processamento de um frame foi de 0,2481895403 segundos. No entanto, para o processamento em tempo real de um vídeo com uma taxa de 24 frames por segundo, o tempo limite de processamento é de 0,0416 segundos por frame ou seja gastou 5,96 vezes mais tempo do que o esperado impossibilitando sua utilização em tempo real. Uma possivel melhoria seria a reescrita da ferramenta em uma linguagem compilada focada em desempenho.






\section[Trabalhos futuros]{Trabalhos futuros}
Alem da melhoria citada em \ref{sec:Pontos fracos e possiveis melhorias} é desejavel para trabalhos futuros um aprimoramento do relatório de modo que contabilize a quantidade de tempo gasto em cada tipo de contração muscular separado por cada iteração do movimento, assim como o tempo medio gasto em cada execução completa alem da implementação de um feedback sonoro fazendo o desuso da parte gráfica da ferramenta de forma que o resultado final não fosse alterado, dando ao usuário um feedback sonóro em tempo real e um feedback visual após o término do teste. 