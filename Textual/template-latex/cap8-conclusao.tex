\chapter{Conclusão}

Neste trabalho, foi desenvolvida e apresentada uma ferramenta a nível de prova de conceito de um sistema de monitoramento do movimento de barra fixa, baseado em testes de aptidão física com base em visão computacional. O objetivo é fornecer ao indivíduo um feedback visual sobre o movimento, a fim de auxiliar na formação da consciência corporal quanto à corretude do movimento.

\section[Problemas encontrados]{Problemas encontrados}\label{sec:Pontos fracos e possiveis melhorias}

A ferramenta apresentada demonstra uma precisão satisfatória na avaliação do movimento de barra fixa, de acordo com a maioria dos parâmetros estabelecidos nos editais citados em \ref{sec:Barra Fixa}. No entanto, uma exceção notável diz respeito à análise da pegada da mão na posição pronada. A tentativa de verificar se a componente $x$ da posição do mindinho da mão direita é menor do que a do indicador da mesma mão, e se a componente $x$ da posição do mindinho da mão esquerda é maior do que a do indicador da mesma mão, revelou-se ineficiente. Isso ocorreu devido a uma imprecisão da ferramenta MidiaPipe na detecção dos pontos de referência 17, 18, 19 e 20 referenciados em \ref{fig:Pontos de referencia usados na aplicacao}, fazendo com que em um determinado momento do vídeo houvesse uma inversão nas referências, o que levou a uma interpretação errônea da pegada.

O tempo médio de processamento de um frame foi de 0,2481895403 segundos. No entanto, para o processamento em tempo real de um vídeo com uma taxa de 24 frames por segundo, o tempo limite de processamento é de 0,0416 segundos por frame, ou seja, gastou 5,96 vezes mais tempo do que o esperado, impossibilitando sua utilização em tempo real.





\section[Trabalhos futuros]{Trabalhos futuros}
É desejável, para trabalhos futuros, um aprimoramento do relatório de modo que contabilize a quantidade de tempo gasto em cada tipo de contração muscular separado por cada iteração do movimento, assim como o tempo médio gasto em cada execução completa. Além disso, sugere-se a implementação de um feedback sonoro, fazendo o desuso da parte gráfica da ferramenta de forma que o resultado final não seja alterado. Isso proporcionaria ao usuário um feedback sonoro em tempo real e um feedback visual após o término do teste.

Também seria relevante elaborar estratégias de testes para avançar com o grau de maturidade da ferramenta e possivelmente transformá-la em um MVP\footnote{MVP (Mínimo Produto Viável) - Uma versão de um novo produto que permite a coleta máxima de aprendizado validado com o mínimo esforço. Em termos simples, é uma versão inicial do produto que inclui apenas as características essenciais para atender às necessidades iniciais dos usuários.}.






